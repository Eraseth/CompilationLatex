\documentclass[a4paper,11pt]{article}
\usepackage[utf8]{inputenc}
\usepackage[francais]{babel}

% --- START - HEADER TO SUPPORT TexSci ALGORITHMS IN LaTeX DOCUMENTS
\usepackage[vlined,ruled]{algorithm2e}
\usepackage{amssymb}
\usepackage{amsmath,alltt}
\newenvironment{texsci}[1]{%
\vspace{1cm}
\renewcommand{\algorithmcfname}{Algorithm}
\begin{algorithm}[H]
\label{#1}
\caption{#1}
\SetKwInOut{Constant}{Constant}
\SetKwInOut{Input}{Input}
\SetKwInOut{Output}{Output}
\SetKwInOut{Global}{Global}
\SetKwInOut{Local}{Local}
}{%
\end{algorithm}
\vspace{1cm}
}
\newcommand{\true}{\mbox{\it true}}
\newcommand{\false}{\mbox{\it false}}
\newcommand{\Boolean}{\{\true,\false\}}
\newcommand{\Integer}{\mathbb{Z}}
\newcommand{\Real}{\mathbb{R}}
% --- END - HEADER TO SUPPORT TexSci ALGORITHMS IN LaTeX DOCUMENTS

\title{TexSci Test Suite -- Test 14\\ Test boucle {\tt for}}
\date{}

\begin{document}
\maketitle

Test de la boucle {\tt for}. Le résultat devrait être : {\tt 42}.

\begin{texsci}{main}
\Local{$a \in \Integer,
        i \in \Integer,
        c \in \Integer$}
\BlankLine
$a \leftarrow 0$\;
\For{$i \leftarrow 0$ \KwTo $20$} {
  $a \leftarrow a + 3$\;
  $a \leftarrow a - 1$\;
  $i \leftarrow i + 1$\;
  $\mbox{printText($"\nValeur de i : "$)}$\;
  $\mbox{printInt($i$)}$\;
  $\mbox{printText($"\n"$)}$\;

  \For{$c \leftarrow 0$ \KwTo $30$} {
    $c \leftarrow c + 1$\;
    $\mbox{printText($"\nValeur de c : "$)}$\;
    $\mbox{printInt($c$)}$\;
    $\mbox{printText($"\n"$)}$\;
  }

}
$\mbox{printInt($a$)}$\;
\end{texsci}

\end{document}
