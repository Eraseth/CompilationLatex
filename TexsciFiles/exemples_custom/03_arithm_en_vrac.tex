\documentclass[a4paper,11pt]{article}
\usepackage[utf8]{inputenc}
\usepackage[francais]{babel}

% --- START - HEADER TO SUPPORT TexSci ALGORITHMS IN LaTeX DOCUMENTS
\usepackage[vlined,ruled]{algorithm2e}
\usepackage{amssymb}
\usepackage{amsmath,alltt}
\newenvironment{texsci}[1]{%
\vspace{1cm}
\renewcommand{\algorithmcfname}{Algorithm}
\begin{algorithm}[H]
\label{#1}
\caption{#1}
\SetKwInOut{Constant}{Constant}
\SetKwInOut{Input}{Input}
\SetKwInOut{Output}{Output}
\SetKwInOut{Global}{Global}
\SetKwInOut{Local}{Local}
}{%
\end{algorithm}
\vspace{1cm}
}
\newcommand{\true}{\mbox{\it true}}
\newcommand{\false}{\mbox{\it false}}
\newcommand{\Boolean}{\{\true,\false\}}
\newcommand{\Integer}{\mathbb{Z}}
\newcommand{\Real}{\mathbb{R}}
% --- END - HEADER TO SUPPORT TexSci ALGORITHMS IN LaTeX DOCUMENTS

\title{TexSci Test Suite -- Test Custom\\ Print en Vrac}
\date{}

\begin{document}
\maketitle

Test de l'affichage d'un entier.

\begin{texsci}{main}
\Local{$i \in \Integer, j \in \Integer, k \in \Real, l \in \Real, m \in \Boolean$}
\BlankLine
$i \leftarrow 21$\;
$j \leftarrow 21$\;
$k \leftarrow 0.5$\;
$l \leftarrow 84.0$\;
$m \leftarrow \true$\;

\tcc{Addition entier}
$\mbox{printText($"\nAddition d'entier (i + j = 42) --> "$)}$\;
$\mbox{printInt($i + j$)}$\;

\tcc{Addition réel}
$\mbox{printText($"\nAddition de réel (10.4 + 0.6 = 11) --> "$)}$\;
$\mbox{printReal($10.4 + 0.6$)}$\;

\tcc{Multiplication entier}
$\mbox{printText($"\nMultiplication d'entier (3 * 5 * 42  = 630). Assignation à i et affichage de i --> "$)}$\;
$i \leftarrow 3 \times 5 \times 42$\;
$\mbox{printInt($i$)}$\;

\tcc{Multiplication réel}
$\mbox{printText($"\nMultiplication de réel (k *  l = 42) --> "$)}$\;
$\mbox{printReal($k \times l$)}$\;

\tcc{Négation}
$\mbox{printText($"\nNégation de valeurs. -i (entier) et -l (réel) --> "$)}$\;
$\mbox{printInt($-i$)}$\;
$\mbox{printText($" "$)}$\;
$\mbox{printReal($-l$)}$\;

\tcc{Priorité}
$\mbox{printText($"\nTest de priorité des opérateurs (+, -, *, / et ()). Test sur 3 * 4 + (5 - 8 / 2) + (2 * -( 1 + 9)). Résultat attendu = -7. Résultat obtenu --> "$)}$\;
$\mbox{printInt($3 \times 4 + (5 - 8 \div 2) + (2 \ast -( 1 + 9))$)}$\;

\tcc{Conversion}
$\mbox{printText($"\nConversion de valeur (int vers float). Ici on printReal de i (entier) * k (réel) (630*0.5 = 315.0) --> "$)}$\;
$\mbox{printReal($i \times k$)}$\;

$\mbox{printText($"\nFin des tests arithmétiques. \n"$)}$\;


\end{texsci}

\end{document}
