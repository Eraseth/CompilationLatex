\documentclass[a4paper,11pt]{article}
\usepackage[utf8]{inputenc}
\usepackage[francais]{babel}

% --- START - HEADER TO SUPPORT TexSci ALGORITHMS IN LaTeX DOCUMENTS
\usepackage[vlined,ruled]{algorithm2e}
\usepackage{amssymb}
\usepackage{amsmath,alltt}
\newenvironment{texsci}[1]{%
\vspace{1cm}
\renewcommand{\algorithmcfname}{Algorithm}
\begin{algorithm}[H]
\label{#1}
\caption{#1}
\SetKwInOut{Constant}{Constant}
\SetKwInOut{Input}{Input}
\SetKwInOut{Output}{Output}
\SetKwInOut{Global}{Global}
\SetKwInOut{Local}{Local}
}{%
\end{algorithm}
\vspace{1cm}
}
\newcommand{\true}{\mbox{\it true}}
\newcommand{\false}{\mbox{\it false}}
\newcommand{\Boolean}{\{\true,\false\}}
\newcommand{\Integer}{\mathbb{Z}}
\newcommand{\Real}{\mathbb{R}}
% --- END - HEADER TO SUPPORT TexSci ALGORITHMS IN LaTeX DOCUMENTS

\title{TexSci Test Suite -- Test Custom\\ Print en Vrac}
\date{}

\begin{document}
\maketitle

Test de l'affichage d'un entier.

\begin{texsci}{main}
\Local{$a \in \Boolean, b \in \Boolean, c \in \Integer, d \in \Integer, e \in \Real, f \in \Real$}
\BlankLine
$a \leftarrow \true$\;
$b \leftarrow \false$\;
$c \leftarrow 2$\;
$d \leftarrow 0$\;
$e \leftarrow 42.5$\;
$f \leftarrow 0.0$\;

\tcc{Le contexte}
$\mbox{printText($"a = true\n"$)}$\;
$\mbox{printText($"b = false\n"$)}$\;
$\mbox{printText($"c = 2\n"$)}$\;
$\mbox{printText($"d = 0\n"$)}$\;
$\mbox{printText($"e = 42.5\n"$)}$\;
$\mbox{printText($"f = 0.0\n"$)}$\;


\end{texsci}

\end{document}
