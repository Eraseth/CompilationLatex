\documentclass[a4paper,11pt]{article}
\usepackage[utf8]{inputenc}
\usepackage[francais]{babel}

% --- START - HEADER TO SUPPORT TexSci ALGORITHMS IN LaTeX DOCUMENTS
\usepackage[vlined,ruled]{algorithm2e}
\usepackage{amssymb}
\usepackage{amsmath,alltt}
\newenvironment{texsci}[1]{%
\vspace{1cm}
\renewcommand{\algorithmcfname}{Algorithm}
\begin{algorithm}[H]
\label{#1}
\caption{#1}
\SetKwInOut{Constant}{Constant}
\SetKwInOut{Input}{Input}
\SetKwInOut{Output}{Output}
\SetKwInOut{Global}{Global}
\SetKwInOut{Local}{Local}
}{%
\end{algorithm}
\vspace{1cm}
}
\newcommand{\true}{\mbox{\it true}}
\newcommand{\false}{\mbox{\it false}}
\newcommand{\Boolean}{\{\true,\false\}}
\newcommand{\Integer}{\mathbb{Z}}
\newcommand{\Real}{\mathbb{R}}
% --- END - HEADER TO SUPPORT TexSci ALGORITHMS IN LaTeX DOCUMENTS

\title{TexSci Test Suite -- Test Custom\\ Print en Vrac}
\date{}

\begin{document}
\maketitle

Test de l'affichage d'un entier.

\begin{texsci}{main}
\Local{$i \in \Integer, j \in \Real, k \in \Boolean$}
\BlankLine
$i \leftarrow 42$\;
$j \leftarrow 42.42$\;
$k \leftarrow \true$\;
\tcc{Affichage de texte}
$\mbox{printText($"\nAffichage de text basique\n"$)}$\;

\tcc{Affichage d'un entier}
$\mbox{printText($"\nAffichage d'un entier : "$)}$\;
$\mbox{printInt($i$)}$\;

\tcc{Affichage d'un réel}
$\mbox{printText($"\nAffichage d'un réel : "$)}$\;
$\mbox{printReal($j$)}$\;


$\mbox{printText($"\n\nFIN du test. Résultat attendu : \nAffichage de text basique\n42\n42.42\n"$)}$\;

\end{texsci}

\end{document}
