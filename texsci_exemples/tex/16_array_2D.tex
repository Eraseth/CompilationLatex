\documentclass[a4paper,11pt]{article}
\usepackage[utf8]{inputenc}
\usepackage[francais]{babel}

% --- START - HEADER TO SUPPORT TexSci ALGORITHMS IN LaTeX DOCUMENTS
\usepackage[vlined,ruled]{algorithm2e}
\usepackage{amssymb}
\usepackage{amsmath,alltt}
\newenvironment{texsci}[1]{%
\vspace{1cm}
\renewcommand{\algorithmcfname}{Algorithm}
\begin{algorithm}[H]
\label{#1}
\caption{#1}
\SetKwInOut{Constant}{Constant}
\SetKwInOut{Input}{Input}
\SetKwInOut{Output}{Output}
\SetKwInOut{Global}{Global}
\SetKwInOut{Local}{Local}
}{%
\end{algorithm}
\vspace{1cm}
}
\newcommand{\true}{\mbox{\it true}}
\newcommand{\false}{\mbox{\it false}}
\newcommand{\Boolean}{\{\true,\false\}}
\newcommand{\Integer}{\mathbb{Z}}
\newcommand{\Real}{\mathbb{R}}
% --- END - HEADER TO SUPPORT TexSci ALGORITHMS IN LaTeX DOCUMENTS

\title{TexSci Test Suite -- Test 16\\ Test tableaux 2D}
\date{}

\begin{document}
\maketitle

Test des tableaux 2D. Le tableau devrait contenir le triangle de Pascal :\\
{\tt   1 }\\
{\tt   1   1 }\\
{\tt   1   2   1 }\\
{\tt   1   3   3   1 }\\
{\tt   1   4   6   4   1 }\\
{\tt   1   5  10  10   5   1 }\\
{\tt   1   6  15  20  15   6   1 }\\
{\tt   1   7  21  35  35  21   7   1 }

\begin{texsci}{main}
\Local{$Pascal \in \Integer^{8 \times 8},
        i \in \Integer,
        j \in \Integer$}
\BlankLine
\For{$i \leftarrow 0$ \KwTo $7$} {
  \For{$j \leftarrow 0$ \KwTo $i$} {
    \eIf{$i = j \vee j = 0$} {
       $Pascal_{i,j} \leftarrow 1$\;
    }{
       $Pascal_{i,j} \leftarrow Pascal_{i-1,j} + Pascal_{i-1,j-1}$\;
    }
  }
}
\For{$i \leftarrow 0$ \KwTo $7$} {
  \For{$j \leftarrow 0$ \KwTo $i$} {
    $\mbox{printInt($Pascal_{i,j}$)}$\;
  }
  \tcc{Fonction d'affichage d'un saut de ligne:}
  $\mbox{println()}$\;}
\end{texsci}

\end{document}
